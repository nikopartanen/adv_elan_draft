\documentclass[]{book}
\usepackage{lmodern}
\usepackage{amssymb,amsmath}
\usepackage{ifxetex,ifluatex}
\usepackage{fixltx2e} % provides \textsubscript
\ifnum 0\ifxetex 1\fi\ifluatex 1\fi=0 % if pdftex
  \usepackage[T1]{fontenc}
  \usepackage[utf8]{inputenc}
\else % if luatex or xelatex
  \ifxetex
    \usepackage{mathspec}
  \else
    \usepackage{fontspec}
  \fi
  \defaultfontfeatures{Ligatures=TeX,Scale=MatchLowercase}
\fi
% use upquote if available, for straight quotes in verbatim environments
\IfFileExists{upquote.sty}{\usepackage{upquote}}{}
% use microtype if available
\IfFileExists{microtype.sty}{%
\usepackage{microtype}
\UseMicrotypeSet[protrusion]{basicmath} % disable protrusion for tt fonts
}{}
\usepackage[margin=1in]{geometry}
\usepackage{hyperref}
\hypersetup{unicode=true,
            pdftitle={Advanced ELAN manipulation and analysis},
            pdfauthor={Niko Partanen},
            pdfborder={0 0 0},
            breaklinks=true}
\urlstyle{same}  % don't use monospace font for urls
\usepackage{natbib}
\bibliographystyle{apalike}
\usepackage{color}
\usepackage{fancyvrb}
\newcommand{\VerbBar}{|}
\newcommand{\VERB}{\Verb[commandchars=\\\{\}]}
\DefineVerbatimEnvironment{Highlighting}{Verbatim}{commandchars=\\\{\}}
% Add ',fontsize=\small' for more characters per line
\usepackage{framed}
\definecolor{shadecolor}{RGB}{248,248,248}
\newenvironment{Shaded}{\begin{snugshade}}{\end{snugshade}}
\newcommand{\KeywordTok}[1]{\textcolor[rgb]{0.13,0.29,0.53}{\textbf{#1}}}
\newcommand{\DataTypeTok}[1]{\textcolor[rgb]{0.13,0.29,0.53}{#1}}
\newcommand{\DecValTok}[1]{\textcolor[rgb]{0.00,0.00,0.81}{#1}}
\newcommand{\BaseNTok}[1]{\textcolor[rgb]{0.00,0.00,0.81}{#1}}
\newcommand{\FloatTok}[1]{\textcolor[rgb]{0.00,0.00,0.81}{#1}}
\newcommand{\ConstantTok}[1]{\textcolor[rgb]{0.00,0.00,0.00}{#1}}
\newcommand{\CharTok}[1]{\textcolor[rgb]{0.31,0.60,0.02}{#1}}
\newcommand{\SpecialCharTok}[1]{\textcolor[rgb]{0.00,0.00,0.00}{#1}}
\newcommand{\StringTok}[1]{\textcolor[rgb]{0.31,0.60,0.02}{#1}}
\newcommand{\VerbatimStringTok}[1]{\textcolor[rgb]{0.31,0.60,0.02}{#1}}
\newcommand{\SpecialStringTok}[1]{\textcolor[rgb]{0.31,0.60,0.02}{#1}}
\newcommand{\ImportTok}[1]{#1}
\newcommand{\CommentTok}[1]{\textcolor[rgb]{0.56,0.35,0.01}{\textit{#1}}}
\newcommand{\DocumentationTok}[1]{\textcolor[rgb]{0.56,0.35,0.01}{\textbf{\textit{#1}}}}
\newcommand{\AnnotationTok}[1]{\textcolor[rgb]{0.56,0.35,0.01}{\textbf{\textit{#1}}}}
\newcommand{\CommentVarTok}[1]{\textcolor[rgb]{0.56,0.35,0.01}{\textbf{\textit{#1}}}}
\newcommand{\OtherTok}[1]{\textcolor[rgb]{0.56,0.35,0.01}{#1}}
\newcommand{\FunctionTok}[1]{\textcolor[rgb]{0.00,0.00,0.00}{#1}}
\newcommand{\VariableTok}[1]{\textcolor[rgb]{0.00,0.00,0.00}{#1}}
\newcommand{\ControlFlowTok}[1]{\textcolor[rgb]{0.13,0.29,0.53}{\textbf{#1}}}
\newcommand{\OperatorTok}[1]{\textcolor[rgb]{0.81,0.36,0.00}{\textbf{#1}}}
\newcommand{\BuiltInTok}[1]{#1}
\newcommand{\ExtensionTok}[1]{#1}
\newcommand{\PreprocessorTok}[1]{\textcolor[rgb]{0.56,0.35,0.01}{\textit{#1}}}
\newcommand{\AttributeTok}[1]{\textcolor[rgb]{0.77,0.63,0.00}{#1}}
\newcommand{\RegionMarkerTok}[1]{#1}
\newcommand{\InformationTok}[1]{\textcolor[rgb]{0.56,0.35,0.01}{\textbf{\textit{#1}}}}
\newcommand{\WarningTok}[1]{\textcolor[rgb]{0.56,0.35,0.01}{\textbf{\textit{#1}}}}
\newcommand{\AlertTok}[1]{\textcolor[rgb]{0.94,0.16,0.16}{#1}}
\newcommand{\ErrorTok}[1]{\textcolor[rgb]{0.64,0.00,0.00}{\textbf{#1}}}
\newcommand{\NormalTok}[1]{#1}
\usepackage{longtable,booktabs}
\usepackage{graphicx,grffile}
\makeatletter
\def\maxwidth{\ifdim\Gin@nat@width>\linewidth\linewidth\else\Gin@nat@width\fi}
\def\maxheight{\ifdim\Gin@nat@height>\textheight\textheight\else\Gin@nat@height\fi}
\makeatother
% Scale images if necessary, so that they will not overflow the page
% margins by default, and it is still possible to overwrite the defaults
% using explicit options in \includegraphics[width, height, ...]{}
\setkeys{Gin}{width=\maxwidth,height=\maxheight,keepaspectratio}
\IfFileExists{parskip.sty}{%
\usepackage{parskip}
}{% else
\setlength{\parindent}{0pt}
\setlength{\parskip}{6pt plus 2pt minus 1pt}
}
\setlength{\emergencystretch}{3em}  % prevent overfull lines
\providecommand{\tightlist}{%
  \setlength{\itemsep}{0pt}\setlength{\parskip}{0pt}}
\setcounter{secnumdepth}{5}
% Redefines (sub)paragraphs to behave more like sections
\ifx\paragraph\undefined\else
\let\oldparagraph\paragraph
\renewcommand{\paragraph}[1]{\oldparagraph{#1}\mbox{}}
\fi
\ifx\subparagraph\undefined\else
\let\oldsubparagraph\subparagraph
\renewcommand{\subparagraph}[1]{\oldsubparagraph{#1}\mbox{}}
\fi

%%% Use protect on footnotes to avoid problems with footnotes in titles
\let\rmarkdownfootnote\footnote%
\def\footnote{\protect\rmarkdownfootnote}

%%% Change title format to be more compact
\usepackage{titling}

% Create subtitle command for use in maketitle
\newcommand{\subtitle}[1]{
  \posttitle{
    \begin{center}\large#1\end{center}
    }
}

\setlength{\droptitle}{-2em}
  \title{Advanced ELAN manipulation and analysis}
  \pretitle{\vspace{\droptitle}\centering\huge}
  \posttitle{\par}
  \author{Niko Partanen}
  \preauthor{\centering\large\emph}
  \postauthor{\par}
  \predate{\centering\large\emph}
  \postdate{\par}
  \date{2017-11-01}

\usepackage{booktabs}

\usepackage{amsthm}
\newtheorem{theorem}{Theorem}[chapter]
\newtheorem{lemma}{Lemma}[chapter]
\theoremstyle{definition}
\newtheorem{definition}{Definition}[chapter]
\newtheorem{corollary}{Corollary}[chapter]
\newtheorem{proposition}{Proposition}[chapter]
\theoremstyle{definition}
\newtheorem{example}{Example}[chapter]
\theoremstyle{definition}
\newtheorem{exercise}{Exercise}[chapter]
\theoremstyle{remark}
\newtheorem*{remark}{Remark}
\newtheorem*{solution}{Solution}
\begin{document}
\maketitle

{
\setcounter{tocdepth}{1}
\tableofcontents
}
\chapter{Info}\label{info}

\section{Course goals}\label{course-goals}

I have been teaching on courses, informal workshops and meetings the
basic use of ELAN and Praat. It seems to me that instructions of how
these programs are used are included in many courses and summer schools,
but it is not maybe so common to move beyond from that. In principle
there is no need for this: once the researcher is familiar with the GUI,
basic usage and search principles, there is not necessarily that much
more to cover in the program itself. I strongly believe that a course of
some week and intensive use of few more is enough to master ELAN.

On the other hand, many courses and handbooks focus into statistical
analysis of linguistic data. I think there is a clear need for more
discussion about the ways how do we get our linguistic data most
effectively into statistical or analytical software we intend to use.
This may sound like a topic that is not worth that much thought, but as
is often said that 80\% of data analysis is data manipulation, the topic
is eventually more central for our daily work than we may think.

There are numerous ways to work programmatically with ELAN files, and
this can be very useful both while producing the new data or analysing
existing files. Although the focus is in ELAN, I will also mention Praat
from time to time. These two programs have somewhat different goals and
niches, which are covered better in its own section. There is also a
very different approach in these tools, since Praat can be very far
manipulated through PraatScript, whereas with ELAN the means available
are bit different.

This course is not an introduction to ELAN or Praat, and it is neither
an introduction to programming. Basic knowledge of R or Python will help
a lot, but brave mind will probably be enough. I think we have to take
as starting point that the majority of us are researchers first, and
programming is neither our job or best skill. However, we can all learn
to pay attention to some basic programming practices that make our work
easier to adopt for others. These includes code comments and version
control, among some other conventions.

The main goal of this course is to help thinking about ways to
automatize some parts of our workflows related to linguistic research.
There are numerous tasks we do which demand hundreds and hundreds of
clicks on mouse, and in all these situations we have to ask: \textbf{are
we spending time with something that can be automatized, or with
something that demands our expert knowledge to be solved?} We have to
maximize the latter, also the time we spend when we try to get into that
level of our work, instead of fighting against the cumbersome manual
workflow where emphasis is easily in unnecessary parts of the task.
Research necessarily is somewhat ``boring'', it is inevitable that we do
some thing thousands of times just to find out that we didn't really
learn that much. If we can find ways to speed up the manual parts of the
process, we have the possibility to wander on even more unnecessary and
poorly rewarding veins of though, which will eventually lead us to
larger questions and their answers.

\section{Course structure}\label{course-structure}

The course materials will be divided into several parts, and which all
are used depends from the length of the course. Some parts can be
skipped, and some can be used only as a reference. For example, the part
about tools contains brief descriptions of the R and Python packages
mainly discussed here with their most commonly used functions, so that
can be a good place to look for help. I have also included there a list
of useful references and links about basic usage of R and Python, just
to get everyone onward.

\begin{enumerate}
\def\labelenumi{\arabic{enumi}.}
\tightlist
\item
  Introduction

  \begin{itemize}
  \tightlist
  \item
    ELAN corpora
  \item
    Goals of this work
  \item
    Available tools
  \end{itemize}
\item
  Structure of the ELAN file

  \begin{itemize}
  \tightlist
  \item
    XML structure
  \item
    How ELAN interacts with its own XML?
  \end{itemize}
\item
  Python examples

  \begin{itemize}
  \tightlist
  \item
    Creating new files with Pympi
  \item
    Manipulating ELAN file with Pympi
  \end{itemize}
\item
  R examples

  \begin{itemize}
  \tightlist
  \item
    Interaction with emuR
  \item
    Interaction between Praat and R
  \item
    Creating new ELAN tiers in R
  \end{itemize}
\item
  Summary
\end{enumerate}

\section{Schedule}\label{schedule}

\begin{itemize}
\tightlist
\item
  Thu 16.11. FRIAS

  \begin{itemize}
  \tightlist
  \item
    09:30-11:00 \textbf{Course} Topic: ELAN corpora \& Goals of
    programmatic workflows \& ELAN file
  \item
    11:00-11:30 Coffee
  \item
    11:30-13:00 \textbf{Course} Topic: Available ecosystem (R, Python,
    ELAN, Praat)
  \item
    13:00-14:00 Lunch (on own expenses)
  \item
    14:00-15:30 \textbf{Course} Topic: Parsing ELAN file + metadata
  \item
    15:30-16:00 Coffee
  \item
    16:00-17:30 \textbf{Course} Topic: Further interaction with ELAN
    corpus in R
  \end{itemize}
\item
  Thu 16.11. University

  \begin{itemize}
  \tightlist
  \item
    18:00--20:00 Guest lecture by
    \href{https://en.wikipedia.org/wiki/Mark_Davies_(linguist)}{Mark
    Davies} ``Examining variation and change in language and culture
    with large online corpora''
  \end{itemize}
\item
  Fr 17.11. FRIAS

  \begin{itemize}
  \tightlist
  \item
    09:30-11:00 \textbf{Course} Topic: ELAN tier manipulation with
    Python
  \item
    11:00-11:30 Coffee
  \item
    11:30-13:00 \textbf{Course} Topic: Examples of tools in interaction
  \item
    13:00-14:00 Lunch (on own expenses)
  \item
    14:00-15:30 \textbf{Course} Topic: Summary
  \end{itemize}
\end{itemize}

\section{Resources}\label{resources}

The whole course exists as an R package, which contains all functions
discussed in the material. It can be installed with:

\begin{Shaded}
\begin{Highlighting}[]
\CommentTok{# comment: fix this later}
\KeywordTok{library}\NormalTok{(devtools)}
\KeywordTok{install_github}\NormalTok{(}\StringTok{'langdoc/adv_elan'}\NormalTok{)}
\end{Highlighting}
\end{Shaded}

Besides the R package, the course repository also contains all Python
code examples from the course. They are maybe also put into a module if
there is time.

The course materials also contain an example ELAN corpus with associated
audio files.

\section{Practical info}\label{practical-info}

I will teach with these materials, or their subset, in a workshop in
Freiburg, around 16.-17. November 2017.

\chapter{Introduction}\label{intro}

\section{Elan and Praat -- strengths and
weaknesses}\label{elan-and-praat-strengths-and-weaknesses}

\section{Linguistic software}\label{linguistic-software}

\section{ELAN corpora}\label{elan-corpora}

\section{Preprocessing}\label{preprocessing}

\subsection{Example of preprocessing
workflow}\label{example-of-preprocessing-workflow}

\section{Analysis workflows}\label{analysis-workflows}

\section{When to write back to ELAN
file}\label{when-to-write-back-to-elan-file}

\section{Annotations as an independent
dataset}\label{annotations-as-an-independent-dataset}

\chapter{Tools}\label{tools}

\section{Git}\label{git}

\section{R}\label{r}

\subsection{\texorpdfstring{\texttt{tidytext}}{tidytext}}\label{tidytext}

\subsection{\texorpdfstring{\texttt{xml2}}{xml2}}\label{xml2}

\section{Python}\label{python}

\subsection{\texorpdfstring{\texttt{pympi}}{pympi}}\label{pympi}

\section{Anaconda}\label{anaconda}

\section{reticulate}\label{reticulate}

\section{PraatScript}\label{praatscript}

\section{XPath}\label{xpath}

\subsection{Examples}\label{examples}

\chapter{ELAN file structure}\label{elan-file-structure}

\section{Minimal file}\label{minimal-file}

\subsection{Participant name
convention}\label{participant-name-convention}

\section{Tier type naming convention}\label{tier-type-naming-convention}

\section{Hierarchies}\label{hierarchies}

\section{Discussion}\label{discussion}

\chapter{Parsing ELAN files to R}\label{parsing-elan-files-to-r}

\section{Why to read ELAN files into
R?}\label{why-to-read-elan-files-into-r}

\section{Parsing with FRelan package}\label{parsing-with-frelan-package}

\chapter{Manipulating ELAN files with
Pympi}\label{manipulating-elan-files-with-pympi}

\chapter{Pympi examples}\label{pympi-examples}

\section{Creating a new ELAN file}\label{creating-a-new-elan-file}

\section{Populating the ELAN file with
content}\label{populating-the-elan-file-with-content}

\section{Merging ELAN files}\label{merging-elan-files}

\chapter{Shiny components}\label{shiny-components}

\section{DT}\label{dt}

\section{Leaflet}\label{leaflet}

\section{ggplot2}\label{ggplot2}

\section{Advantages and disadvantages of
Shiny}\label{advantages-and-disadvantages-of-shiny}

\chapter{Example: Interaction with
emuR}\label{example-interaction-with-emur}

\section{Procedure}\label{procedure}

\chapter{Example: Interaction with
Praat}\label{example-interaction-with-praat}

\section{Research questions}\label{research-questions}

\section{Implementation}\label{implementation}

\section{Shiny application}\label{shiny-application}

\section{Observations}\label{observations}

\section{Exercise}\label{exercise}

\chapter{Example: Concordances and
map}\label{example-concordances-and-map}

\section{Use}\label{use}

\chapter{Example: Preprocessing
workflow}\label{example-preprocessing-workflow}

\section{From points to polygon}\label{from-points-to-polygon}

\section{based on this:}\label{based-on-this}

\section{\texorpdfstring{\url{https://gis.stackexchange.com/questions/222978/lon-lat-to-simple-features-sfg-and-sfc-in-r}}{https://gis.stackexchange.com/questions/222978/lon-lat-to-simple-features-sfg-and-sfc-in-r}}\label{httpsgis.stackexchange.comquestions222978lon-lat-to-simple-features-sfg-and-sfc-in-r}

\chapter{Final Words}\label{final-words}

\chapter{Placeholder}\label{placeholder}

\bibliography{packages.bib,book.bib}


\end{document}
